\documentclass[pdftex,11pt,letter]{article}
\input{include.tex}

\title{\textbf{TASS: A Toolkit for Aircraft Sizing and Synthesis}}
\author{Komahan Boopathy~~\url{komahan@gatech.edu}} \date{\today}
\begin{document}

\maketitle
\vspace{-0.25in}
\rule{\linewidth}{2pt}

\begin{abstract}
A Toolkit for Aircraft Sizing and Synthesis (TASS), capable of performing sizing and synthesis calculations for conceptual design of aircraft is developed in Matlab~\cite{MATLAB}. TASS implements energy-based constraint and weight fraction approach for the mission sizing analyses.  The performance of TASS is benchmarked against the known metrics of a transonic jet fighter aircraft F-86L Sabre (Sabrejet).
\end{abstract}

\section{Introduction and Objectives}


The purpose of this effort is to develop a toolkit in Matlab to perform sizing and synthesis of aircraft configurations at the level of conceptual design. The tool is named ASSIST. 

 that applies the concepts in prelimi of energy based constraint~\cite{RaymerText}.

\bibliography{KomahannoVol}
\bibliographystyle{plain}	

\end{document}

\begin{figure}[h!]
	\centering
	\includegraphics[scale=0.85]{prob1_ques.pdf}
	\label{fig:prob1_ques}
\end{figure}

sizing & synthesis,

energy-based
constraint analysis method

weight fraction approach to the mission sizing analysis

An analysis tool
capable of fully implementing sizing & synthesis calculations must be created and validated against existing data.

benchmark your tool against the known performance of a reference aircraft, provided in this project
description

This section should also introduce the concept under study and briefly summarize the mission and
requirements.


predicted takeoff gross weight, T/W, and W/S, calculate the predicted maximum required thrust
at sea level and the predicted wing area. Compare these predicted values to the known values within the Standard
Aircraft Characteristics.



%%

\cite{MavrisNotes}
~\cite{NicolaiText}
~\cite{FieldingText}
~\cite{HoweText}
~\cite{RaymerText}
~\cite{Raymer2004}
