\usepackage{appendix} %http://www.tex.ac.uk/cgi-bin/texfaq2html?label=appendix
\usepackage{cancel}
\usepackage{amsmath,amssymb,latexsym,float,epsfig}
\usepackage{framed,color,url,fancybox,fullpage,booktabs,subfigure,wrapfig,chngpage,setspace}

\usepackage{amsfonts,caption}
\usepackage{hyperref}
\usepackage{graphicx}
\usepackage{color,epsfig}
\usepackage{bm}
\usepackage{enumerate}
\usepackage{amsmath, amssymb, graphics, setspace,mathtools}
\usepackage{float}
\usepackage[section]{placeins} % places floats within the section
\usepackage[superscript]{cite}


\newcommand{\ssection}[1]{\section[#1]{\centering\normalfont\scshape #1}}
\newcommand{\ssubsection}[1]{\subsection[#1]{\raggedright\normalfont\itshape #1}}
\newcommand\norm[1]{\left\lVert#1 \right\rVert}
\newcommand{\e}{\mathrm{e}}
\newcommand{\pderiv}[2]{\frac{\partial #1}{\partial #2}}
\numberwithin{equation}{section}
\numberwithin{figure}{section}
\usepackage{epstopdf,fancyvrb,cite,hyperref,jvlisting}
%\usepackage[options]{mcode}

\usepackage{pdflscape}%http://texblog.org/2007/11/10/landscape-in-latex/

\newcommand{\ul}[1]{\underline #1}

% Define commands 
\newcommand{\half}{\ensuremath{\frac{1}{2}}}
\newcommand{\bea}{\begin{eqnarray}}
\newcommand{\eea}{\end{eqnarray}}
\newcommand{\beq}{\begin{equation}}
\newcommand{\eeq}{\end{equation}}
\newcommand{\bdm}{\begin{displaymath}}
\newcommand{\edm}{\end{displaymath}}

\newcommand{\etal}[0]{{\em et al.}}
\newcommand{\etc}[0]{{\em etc.}}
\newcommand{\ie}[0]{{\em i.e.,}}


\newcommand{\pd}[2]{\dfrac{\partial #1}{\partial #2}}
\newcommand{\pf}[2]{\dfrac{d #1}{d #2}}
\newcommand{\pdt}[2]{\dfrac{\partial^2 #1}{\partial #2^2}}
\newcommand{\pft}[2]{\dfrac{d^2 #1}{d #2^2}}
\newcommand{\pdtno}[2]{\dfrac{\partial^2 #1}{\partial #2}}
\newcommand{\pdd}[3]{\dfrac{\partial^2 #1}{\partial #2 \partial #3}}
\newcommand{\pff}[3]{\dfrac{d^2 #1}{d #2 d #3}}


\renewcommand\floatpagefraction{0.99}
\renewcommand\topfraction{0.99}
\renewcommand\bottomfraction{0.99}
\renewcommand\textfraction{0.0}

\usepackage{titling}
\setlength{\droptitle}{-1in}   % This is your set screw

% For \url{SOME_URL}, links SOME_URL to the url SOME_URL
\providecommand*\url[1]{\href{#1}{#1}}
% Same as above, but pretty-prints SOME_URL in teletype fixed-width font
\renewcommand*\url[1]{\href{#1}{\texttt{#1}}}

% For \email{ADDRESS}, links ADDRESS to the url mailto:ADDRESS
\providecommand*\email[1]{\href{mailto:#1}{#1}}
